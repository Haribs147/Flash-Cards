% !TEX program = pdflatex
% !TeX encoding = utf8
% !TeX spellcheck = pl-PL

%%%%%%%%%%%%%%%%%%%%%%%%%%%%%%%%%%%%%%%%%%%%%%%%%%%%%%%%%%%%%%%%%%%%%%%%%%%
% Wybierz rodzaj pracy dyplomowej oraz wydział
% Pick thesis type and faculty
%%%%%%%%%%%%%%%%%%%%%%%%%%%%%%%%%%%%%%%%%%%%%%%%%%%%%%%%%%%%%%%%%%%%%%%%%%%
\documentclass[thesis=inz,faculty=ee]{EE-dyplom} 

% thesis=[inz|mgr|bsc|msc]
%  * inz - praca inżynierska
%  * mgr - praca magisterska
%  * bsc - bachelor thesis
%  * msc - master thesis

% Skróty nazw wydziałów zgodne z domenami internetowymi
% Abbreviations of Faculties according to Internet subdomains
% faculty=[
%	arch,
%	gik,
%	ee,
%	wip
%	]
\usepackage{placeins}
\usepackage{dirtree}

\usepackage{listings}
\usepackage{xcolor}
\definecolor{ac-red}{rgb}{0.8,0,0}       % Keywords (like 'def', 'return')
\definecolor{ac-blue}{rgb}{0,0,0.8}      % Functions/Types
\definecolor{ac-purple}{rgb}{0.5,0,0.5}  % Strings
\definecolor{ac-gray}{rgb}{0.5,0.5,0.5}  % Comments
\definecolor{ac-black}{rgb}{0,0,0}       % Standard text


%%%%%%%%%%%%%%%%%%%%%%%%%%%%%%%%%%%%%%%%%%%%%%%%%%%%%%%%%%%%%%%%%%%%%%%%%%%
% Konfiguracja - do personalizacji
% Configuration - to be personalized
%%%%%%%%%%%%%%%%%%%%%%%%%%%%%%%%%%%%%%%%%%%%%%%%%%%%%%%%%%%%%%%%%%%%%%%%%%%
\instytut{Instytut Elektrotechniki Teoretycznej i Systemów Informacyjno-Pomiarowych}
\kierunek{Informatyka Stosowana}
\specjalnosc{Inżynieria Oprogramowania}
\title{Aplikacja internetowa do zarządzania fiszkami}
\engtitle{A web app for managing flashcards }
\album{325473}
\author{inż. Michał Jagodziński}
\promotor{inż. Marek Wdowiak}
\date{2025}
\longdate{2025-12-31}

%%%%%%%%%%%%%%%%%%%%%%%%%%%%%%%%%%%%%%%%%%%%%%%%%%%%%%%%%%%%%%%%%%%%%%%%%%%
% Streszczenie pracy i abstract.
% In case of thesis in English swap the order - English version goes first.
%%%%%%%%%%%%%%%%%%%%%%%%%%%%%%%%%%%%%%%%%%%%%%%%%%%%%%%%%%%%%%%%%%%%%%%%%%%
\streszczeniepracy{
To jest streszczenie. To jest trochę za krótkie, jako że powinno zająć całą stronę.

Streszczenie to nie to samo co wstęp. Streszczenie przedstawia informacje o tym, co się znajduje w tekście poszczególnych rozdziałów danej pracy podczas gdy ,,Wstęp'' służy przedstawieniu zagadnienia podjętego w~pracy. Streszczenie prezentuje aktualny stan pracy dyplomowej, której pisanie zostało zakończone a~więc należy w~nim posługiwać się czasem teraźniejszym.

} % koniec streszczenia

\slowakluczowe{A, B, C}

\thesisabstract{
This is abstract. This one is a little too short as it should occupy the whole page.

An abstract is not the same as an introduction. The abstract provides information about what is contained in the text of specific chapters of a given work, whereas the introduction serves to present the topic addressed in the work. The abstract presents the current state of the diploma thesis, the writing of which has been completed, and therefore it should be written in the present tense.
} % end of abstract

\thesiskeywords{X, Y, Z}

%%%%%%%%%%%%%%%%%%%%%%%%%%%%%%%%%%%%%%%%%%%%%%%%%%%%%%%%%%%%%%%%%%%%%%%%%%%
% Tu zaczyna się dokument
% Here is the beginning of the document
%%%%%%%%%%%%%%%%%%%%%%%%%%%%%%%%%%%%%%%%%%%%%%%%%%%%%%%%%%%%%%%%%%%%%%%%%%%
\begin{document}
    % Strony nagłówkowe - musi być
    % Headers - must exist
    \frontpages
    
    % Tekst pracy - osobne pliki do edycji
    % Thesis text - separate files for edition

    % 1. wstep
    % 1.1 Wprowadzenie do problematyki
    %     opisał bym na pół/jedną stronę bym opisał czemu moja aplikacja powstała. Na przykład: "W ciągu życia każdy człowiek nieustannie
    %     pragnie pogłebiać swoją wiedzę. Jest to jednak trudne albowiem w ciągu dnia brakuje czasu, więc ludzi muszą jakoś wypełniać luki
    %     czasu które spędzają na przykład w drodze tramwajem na studia lub do pracy, albo ... . W erze internetu jest na to proste rozwiazanie,
    %     można przełożyć wiedzę na proste fiszki łatwe do wykorzystwania za pomocą telefonu lub komputera itp. itd. tak na pół/jedną strone.
    % 1.2 Analiza porównawcza // Pytanie czy tutaj po prostu zrobić wprowadzenie do problemu, i cel i zakres pracy, a analize porównawczą potem
    %     Na więcej stron czy tak jak teraz:

    %     Tutaj bym opisał  na stronę że obecne rozwiązania są grosza warte. Za quizleta trzeba płacić,
    %         nie ma folderów, udostępniania itp. Wiele niepotrzebnych funkcji, cięzkie UI w którym sie łatwo zgubić. 
    %         Anki na przykład z drugiej strony ma jeszcze gorsze UI i w sumie to te same problemy.
    % 1.3 Cel i zakres pracy
    %     Tutaj chciałbym **opisać cel pracy** Czyli np. celem pracy jest zaprojektowanie i stworzenie
    %     internetowej aplikacji do zarządzania fiszkami. Aplikacja ta musi być przyjazna dla użytkownika,
    %     bezpieczna i przede wszystkim wydajna. Użytkownicy mogą używać takiej aplikacji do wspomagania procesów
    %     nauczania i zapamiętywania.

    %     Potem opisałbym zakres pracy, czyli szczegółowo określiłym bym zakres funkcjonalny systemu.
    %     Aplikacja musi zawierać następujące moduły:
    %     - Uwierzytelniania użytkowników
    %     - Zarządzania zestawami fiszek
    %     - Zarządzania zasobami
    %     - Wyszukiwania i przegladania fiszek
    %     - Oceniania i kommentowania
    %     - Udostępniania 

    %     Opisałbym każdy z modółów w 2-3 zdaniach, np.
    %     **Uwierzytelnianie** - Użytkownicy mają możliwość rejstacji i logowania do aplikacji za pomocą swojego maila i bezpiecznego hasła.
    %     **Zarządzanie zestawami fiszek** - Użytkownicy mają możliwość tworzenia fiszek w formacie WYSIWYG, mogą wstawiać zdjęcia,
    %     nazwę zestawu, opis i zarządzać jego dostępnością (publiczny/prywatny)
    %     **Zarządzanie zasobami** - Użytkownicy mogą tworzyć i usuwać foldery. Foldery mają chierachiczną strukturę, 
    %     dzięki czemu można robić podfoldery przemieszczając je między menu okruszkowemu
    %     **Wyszukiwania i przegladania fiszek** - Użytkownicy bez względu na stan uwierzytelnienia mają dostęp do strony publicznej, zawierającej
    %     najnowsze, najpopularniejsze i najbardziej lubiane publiczne zestawy fiszek. Każdy użytkownik może także przeszukiwać bazę publicznych zestawów fiszek
    %     za pomocą wyszukiwarki tekstowej.

    %     **Nie wiem czy ten podrozdział zamienić na dwa podrozdziały, czyli 1.1 cel i 1.2 zakres pracy?**

    % 1.4 Przegląd stosu technologicznego
    %     Po krótce opisać 1-2 zdania opisane technologie tak jak to z panem rozmawialiśmy
    %     Nie opisywać czym jest python, tylko może bardziej że fastapi, i że szybkie jeśli są długie operacje I/O itp.
    %     Że do sanityzacji nh3 itp.
    %     Opisać mikroservicy w 2-3 zdaniach i jak to pospinałem (elastic search, minio, kibana z apm serwerem)
    %     Opisać że wykorzystałem react + redux + ts do frontendu a do WYSIWYG użyłem TIPTAP
    
    % 1.4 Struktura pracy (Opis zawartości kolejnych rozdziałów [Nie wiem czy to potrzebne?, zapytać się] promotora)

    % 2. Projekt systemu
    % 2.1 Wymagania funkcjonalne
    % np.:
    % - Uwierzytelnianie (rejestracja, logowanie, wylogowanie, ochrona CSRF).
    % - Zarządzanie materiałami (tworzenie folderów, zestawów fiszek, zmiana nazwy, usuwanie, przeciąganie).
    % - Tworzenie i edycja fiszek (obsługa tekstu sformatowanego, obrazów, siatek obrazów).
    % - Nauka (przeglądanie fiszek, odwracanie, tasowanie, odwracanie awers/rewers).
    % - Funkcje społecznościowe (udostępnianie (z uprawnieniami viewer/editor), akceptacja/odrzucenie zaproszenia, głosowanie, komentowanie zagnieżdżone).
    % - Wyszukiwanie (globalne wyszukiwanie zestawów).
    % 2.2 Wymagania niefunkcjonalne
    % np. 
    % - Wydajność: Szybkie odpowiedzi API (uzasadnia FastAPI), szybkie wyszukiwanie (uzasadnia Elasticsearch).
    % - Skalowalność: Możliwość obsługi dużej liczby obrazów (uzasadnia Minio) i użytkowników (architektura Docker).
    % - Bezpieczeństwo: Ochrona danych użytkownika (JWT, httponly cookies, hashowanie haseł z pepperem, sanityzacja HTML).
    % - Niezawodność: Odporność na awarie poszczególnych komponentów (architektura mikrousług).
    % - Użyteczność: Nowoczesny i responsywny interfejs (uzasadnia React).
    % 2.3 Architektura systemu i przepływ danych
    %     Diagram architektury pokazujący połączenie miedzy elastic searchem, minio, kibana + apm server
    %     Może jakis opis przepływu danych dla kluczowych operacji np. dodawanie obrazka albo dodawanie tagów
    % 2.4 Interfejs Użytkownika (Frontend)
    %     Opisać kluczowe elementy, np. stronę publiczną z filtrowaniem po danym okresie czasu, eksplorator plików zrobiony z breadcrumbsumi.
    %     Opisać tiptap że jest headless i musiałem samemu dodać obrazki itp. Opisać komentarze, że mogą być zagnieżdżane, lajkowane itp
    % 2.5 Serwer (Backend)
    %     Opisanie że backend zrobiony został jako REST API opisać co to jest i jak to zostało zaimplementowane, opisać modele pydantic itp.
    % 2.6 Architektura bazy danych
    %     Zamieścić diagram ERD (Entity-Relationship-Diagram)
    %     Omówić najciekawsze i najważniejsze relacje, takie jak np.
    %     Polimorficzna struktura materials (jak item_type, parent_id i linked_material_id tworzą system plików i udostępniania)
    %     Relacja 1-do-1 między materials a flashcard_sets
    %     Implementacja głosowania (votes jako tabela polimorficzna votable_type, votable_id)
    %     Implementacja zagnieżdżonych komentarzy (wyjaśnienie użycia parent_comment_id oraz ltree z changelog.xml)
    % 3.  Implementacja systemu

    % 3.1 Implementacja API
    
    % 3.1.1 Moduł Uwierzytelniania
    % 3.1.2 Zarządzanie Materiałami
    % 3.1.3 Moduł Funkcji Społecznościowych (Like'owanie, komentowanie)
    % 3.1.5 Moduł wyszukiwania i strony publicznej
    % 3.1.6 Integracja z zewnętrznymi serwisami
    
    % 3.2 Implementacja interfejsu użytkownika

    % 3.2.1 Zarządzanie stanem
    % 3.2.2 Edytor WYSIWYG (TipTap że jest headless i że sam musiałem obrazki zaimplementować jak przesyła dane)

    % 3.3 Implementacja mechanizmów bezpieczeństwa
    
    % 3.3.1 No własnie nie wiem czy tu opisać ten moduł uwierzytelniania czy wcześniej, tu jeszcze by były csrf, sanityzacja obrazów i tekstu
    % 3.4 Zarządzanie migracjami bazy danych (Tu bym opisał Liquibase) 
        
    % 4. Wdrożenie i utrzymanie systemu

    % 4.1 Konteneryzacja
    % 4.2 Strategia obserwowalności z openTelemetry

    % 5. Obszary do rozwoju
    % 6. Podsumowanie
    \pagenumbering{arabic}
    \setcounter{page}{1}
    \pagestyle{empty}
    \chapter{Wstęp}
\label{ch:wstep}

Nauka jest nieodłącznym procesem życia każdego człowieka. Szybki rozwój nowych technologii w ostatnich dekadach znacząco zmienił podejście 
społeczeństwa do edukacji. Nastąpił gwałtowny przeskok z papierowych, tradycyjnych nośników wiedzy na łatwo dostępne cyfrowe rozwiązania. 
Proces ten, określany mianem cyfrowej transformacji edukacji, wymusił na społeczeństwie poszukiwanie nowych, 
bardziej efektywnych metod zarządzania wiedzą i narzędzi do jej gromadzenia.

Współczesny uczeń musi być w stanie przyswoić dużą ilość wiedzy w ograniczonym czasie. Tradycyjne metody nauki, opierające się na tworzeniu 
monotonnych, ręcznie zapisanych notatek, mogą okazać się nieskuteczne w zależności od rodzaju przyswajanego materiału. W przypadku nauki 
słówek, definicji czy dat, coraz większą popularność zyskują metody dydaktyczne polegające na aktywnym powtarzaniu nabytej wiedzy. Jednym z 
najczęściej używanych i najskuteczniejszych sposobów jest nauka z samodzielnie przygotowanych fiszek. Manualny proces wykonywania fiszek ma 
jednak też swoje wady. Jest on dosyć czasochłonny, a nauka z wyciętych kartek jest niepraktyczna i niewygodna.

Odpowiedzą na te problemy, jest cyfryzacja zestawów fiszek oraz dostarczenie narzędzi usprawniających ich zarządanie. Głównym celem niniejszej 
pracy jest zaprojektowanie i zaimplementowanie nowoczesnej aplikacji internetowej, która eliminuje ograniczenia fizycznych zestawów fiszek.
Stworzony system oferuje zaawansowany edytor do tworzenia fiszek oraz możliwości kategoryzacji i współdzielenia zgromadzonej wiedzy.

Realizacja tak zdefiniowanego celu objęła pełny cykl wytwarzania oprogramowania (podejście End-to-End), od analizy wymagań i projektu systemu, 
poprzez implementację i przygotowanie środowiska uruchomieniowego oraz telemetrii. Rezultatem jest kompletny system oparty na modelu 
klient-serwer. Warstwa prezentacji została zaimplementowana jako aplikacja typu Single Page Application przy użyciu biblioteki React, 
natomiast serwer w języku Python przy wykorzystaniu frameworka FastAPI. Całość została skonteneryzowana w celu zapewnienia spójnosci środowiska 
oraz ułatwienia przyszłego skalowania i wdrażania aplikacji

\section{Psychologiczne aspekty zapamiętywania}

Aby wyjaśnić, dlaczego stosowanie fiszek w edukacji jest tak wydajne, należy odwołać się do podstawowych badań z zakresu psychologii poznawczej. 
Kluczowe jest zrozumienie różnicy między biernym, a aktywnym przyswajaniem wiedzy. Tradycyjne bierne metody dydaktyczne, polegające na 
wielokrotnym czytaniu notatek lub podręczników, często prowadzą do zjawiska iluzji kompetencji (ang. illusion of competence). 
Zjawisko to polega na fałszywym przekonaniu o umiejętności samodzielnego odtworzenia materiału z głowy, po jedynie kilkukrotnym 
jego przeczytaniu \cite{brown2014make}. W rzeczywistości jednak cała wiedza często pozostaje jedynie w pamięci krótkotrwałej, czyniąc ją ulotną.

Fiszki pozwalają na zmianę strategii nauki na proces aktywny, w którym dochodzi do długotrwałej retencji wiedzy, eliminując problem 
pamięci krótkotrwałej. Podejście to polega na dwóch kluczowych mechanizmach kogniwistycznych:
\begin{itemize}
    \item \textbf{Aktywne przywołanie} -- Metoda nauki polegająca na próbie wydobycia informacji z pamięci poprzez testowanie swojej 
    wiedzy na każdym możliwym etapie nauki bez używania podpowiedzi. Fiszki nadają się do tego perfekcyjnie, ponieważ zawsze prezentują 
    tylko jedną stronę (awers), co zmusza użytkownika do samodzielnego sformułowania odpowiedzi przed jej weryfikacją (zobaczeniem rewersu).
    Badania pokazują, że ten wysiłek kogniwistyczny zapewnia znacznie trwalszą retencję wiedzy, niż proste wielokrotne studiowanie danego 
    materiału, co udowodniono w eksperymentach Roedigera i Karpicke \cite{roediger2006test}.
    \item \textbf{Metapoznanie i samoocena} -- Fiszki wymuszają na uczącym się ciągłą weryfikację stanu swojej wiedzy. W momencie 
    odsłonięcia rewersu karty, użytkownik dokonuje weryfikacji poprawności swojej odpowiedzi. Pozwala to na zidentyfikowanie 
    tematów, które nie zostały jeszcze w pełni opanowane i skierowanie na nich większej uwagi. Badania Kornella i Bjorka 
    \cite{kornell2008optimizing} wykazują, że zdolność do trafnej oceny własnej wiedzy jest kluczowa dla optymalizacji 
    czasu nauki, lecz obarczona ryzykiem błędu w samoocenie. Struktura fiszek minimalizuje ten błąd poprzez atomizację materiału, 
    tym samym ułatwiając trafną ocenę wiedzy na poziomie pojedynczych pojęć.
\end{itemize}

\section{Analiza istniejących rozwiązań}
Fiszki ogromnie zyskały na popularności wraz z coraz większą dostępnością internetu i rozwojem technologicznym.
Na rynku istnieje wiele aplikacji oferujących szeroki zakres narzędzi służących do zarządzania i 
wspomagania procesu zapamiętywania. Wiele z nich opiera się na cyfrowej implementacji fiszek. 
W celu zdefiniowania wymagań projektowych dokonano analizy trzech głównych rozwiązań dostępnych na rynku:
Anki, Quizlet oraz Brainscape.

\subsection{Anki}
Anki to darmowe oprogramowanie typu open-source, które istnieje na rynku już od wielu lat. Projekt oferuje darmową aplikację 
desktopową oraz mobilną na system Android, lecz niestety płatną wersję na IOS. Jego główną zaletą jest 
implementacja zaawansowanego algorytmu powtórek w odstępach czasowych (Spaced Repetition System - SRS).
\begin{itemize}
    \item \textbf{Zalety:} Duża elastyczność w konfiguracji algorytmu SRS, obsługa wtyczek tworzonych przez społeczność i możliwość 
    uczenia się na wielu platformach offline.
    \item \textbf{Wady:} Wszystkie platformy posiadają bardzo nieintuicyjny i przestarzały interfejs graficzny. 
    Krzywa uczenia się obsługi programu jest wysoka, a edytor fiszek mimo dużej liczby funkcjonalności wymaga znajomości 
    HTML/CSS do zaawansowanego formatowania. Ponadto synchronizacja danych między urządzeniami nie jest natywna i wymaga 
    dodatkowej konfiguracji w serwisie AnkiWeb, a system organizacji materiałów jest mało czytelny.
\end{itemize}

\subsection{Quizlet}
Quizlet to komercyjne rozwiązanie, które zdobyło popularność, dzięki intuicyjnemu interfejsowi i dostępności z poziomu 
przeglądarki oraz aplikacji mobilnej. Firma kładzie większy nacisk na implementację różnorodnych form nauki (np. poprzez 
gry) niż na surowe algorytmy powtórek.
\begin{itemize}
    \item \textbf{Zalety:} Nowoczesny i estetyczny interfejs użytkownika, ogromna baza publicznych zestawów fiszek stworzonych 
    przez społeczność Quizleta oraz wysoka dostępność na urządzeniach mobilnych i desktopowych bez konieczności instalacji oprogramowania.
    \item \textbf{Wady:} Model biznesowy typu freemium, polegający na przyciąganiu klientów poprzez oferowanie podstawowych funkcjonalności bezpłatnie, 
    jednocześnie wymagając płatnej subskrypcji za użytkowanie rozszerzonych funkcji systemu. Przykładem funkcjonalności premium jest możliwość wstawiania multimediów 
    do fiszek, czy nauka w trybie długoterminowym. Dodatkowo edytor treści fiszek jest uproszczony i nie pozwala na zaawansowane formatowanie tekstu lub wstawianie wielu zdjęć.
\end{itemize}

\subsection{Brainscape}
Brainscape jest dostępne jako aplikacja mobilna oraz webowa. Korzysta ona z zaawansowanej metody metapoznania i samooceny. W przeciwieństwie do 
standardowych alogrytmów, użytkownik zamiast wyboru odrzucenia/zostawienia fiszki, musi samodzielnie ocenić stopień znajomości odpowiedzi 
w skali od 1 do 5, co bezpośrednio wpływa na czas do następnej powtórki.
\begin{itemize}
    \item \textbf{Zalety:} Wymusza na użytkowniku ciągłą samoocenę, co znacznie przyśpiesza naukę. Aplikacja jest przejrzysta i prosta w obsłudze, 
    nastawiona na szybkie tworzenie prostych fiszek, a edytor treści pozwala na wstawianie plików dźwiękowych i graficznych. 
    \item \textbf{Wady:} Podobnie jak Quizlet, system opiera się na płatnych subskrypcjach, gdzie darmowa wersja posiada ograniczone funkcjonalności 
    (np. brak statystyk i możliwości wstawiania plików multimedialnych). Edytor treści nie oferuje funkcji tworzenia niestandardowych układów grafik i 
    bardziej zaawansowanego formatowania tekstu.
\end{itemize}

\subsection{Podsumowanie analizy}
Przeprowadzona analiza pozwoliła na zidentyfikowanie luki na rynku aplikacji do fiszek. Istnieje realne zapotrzebowanie 
na system, który będzie wspomagał użytkownika nie tylko w nauce, ale także w samym zarządzaniu zestawami fiszek, łącząc zalety 
powyższych aplikacji i eliminując ich główne wady. Tabela \ref{tab:porownanie_rozwiazan} ukazuje zestawienie porównawcze 
analizowanych systemów z aplikacją będącą przedmiotem niniejszej pracy. % POPRAW TO ZDANIE PLS

\begin{table}[htbp]
    \centering
    \caption{Porównanie analizowanych rozwiązań z projektowanym systemem}
    \label{tab:porownanie_rozwiazan}
    \footnotesize % Zmniejszenie czcionki, aby tabela była czytelna
    \begin{tabularx}{\textwidth}{@{} p{2.2cm} X X X X @{}}
        \toprule
        \textbf{Cecha} & \textbf{Anki} & \textbf{Quizlet} & \textbf{Brainscape} & \textbf{Projektowana Aplikacja} \\
        \midrule
        
        \textbf{Model dystrybucji} & 
        Open Source (Desktop/Android), Płatny (iOS) & 
        Komercyjny (Freemium) & 
        Komercyjny (Freemium) & 
        \textbf{Open Source} \\ 
        \addlinespace % Dodatkowy odstęp dla czytelności
        
        \textbf{Dostępność} & 
        Desktop, Mobile (Offline first) & 
        Web, Mobile & 
        Web, Mobile & 
        \textbf{Web} \\ 
        \addlinespace
        
        \textbf{Interfejs (GUI)} & 
        Przestarzały, nieintuicyjny & 
        Nowoczesny & 
        Nowoczesny & 
        \textbf{Nowoczesny} \\ 
        \addlinespace
        
        \textbf{Organizacja danych} & 
        System talii (Decks) & 
        Listy i foldery (Płaska struktura) & 
        Klasy i talie & 
        \textbf{Pełny eksplorator plików (Zagnieżdżone foldery)} \\ 
        \addlinespace
        
        \textbf{Edytor treści} & 
        HTML/CSS (Wymaga wiedzy tech.) & 
        Uproszczony & 
        Podstawowy & 
        \textbf{Zaawansowany WYSIWYG (TipTap)} \\ 
        \addlinespace
        
        \textbf{Wsparcie AI} & 
        Brak (ew. wtyczki 3rd party) & 
        Płatne (Q-Chat) & 
        Brak & 
        \textbf{Automatyczne tagowanie (Gemini API)} \\ 
        
        \bottomrule
    \end{tabularx}
\end{table}

Projektowania aplikacja ma za zadanie wypełnić zidentyfikowaną na rynku niszę. Implementowane rozwiązanie webowe charakteryzuje się 
nowoczesnym i intuicyjnym interfejsem graficznym, zapewniającym duży komfort pracy. Kluczowym elementem jest zaawansowany edytor tekstu, 
który pozwala na tworzenie niestandardowych układów obrazów oraz bogate formatowanie tekstu bez konieczności znajomości kodu HTML czy CSS.
Oprócz tego system kładzie duży nacisk na aspekty społecznościowe, takie jak komentarze czy system głosowania, przekształacjąc samotny 
proces nauki w doświadczenie oparte na współpracy i wymianie wiedzy. Całość dopełnia łatwy w obsłudze eksplorator plików z obsługą 
tworzenia zagnieżdżonych folderów, co ułatwia użytkownikom centralizację wiedzy i zarządzanie wszystkimi materiałami. 


% Współczesna inżynieria oprogramowania oraz dynamiczny rozwój społeczeństwa informacyjnego stawiają przed jednostką wyzwanie 
% ciągłego podnoszenia kwalifikacji. Proces uczenia się, niegdyś ograniczony do murów szkolnych i akademickich, stał się obecnie 
% procesem trwającym całe życie (\textit{lifelong learning}). W obliczu rosnącej objętości danych i specjalistycznej wiedzy, 
% tradycyjne metody notowania i przyswajania informacji, oparte na linearnym czytaniu tekstu, często okazują się niewystarczające. 
% Wymusza to poszukiwanie narzędzi, które nie tylko gromadzą wiedzę, ale także aktywnie wspierają proces jej utrwalania.

% Transformacja cyfrowa edukacji przeniosła ciężar organizacji materiałów dydaktycznych z fizycznych notatników na aplikacje 
% internetowe i mobilne. Zmiana ta otworzyła nowe możliwości w zakresie dostępności wiedzy, współdzielenia zasobów oraz 
% wykorzystania multimediów w procesie poznawczym. Współczesne systemy informatyczne pozwalają na odejście od statycznych 
% modeli danych na rzecz interaktywnych rozwiązań, które dostosowują się do potrzeb użytkownika, umożliwiając naukę w 
% dowolnym miejscu i czasie.

% Jedną z najskuteczniejszych metod systematyzowania wiedzy, szczególnie w obszarach wymagających zapamiętania dużej liczby definicji, terminologii czy wzorców, jest metoda fiszek (\textit{flashcards}). Opiera się ona na mechanizmie aktywnego przywoływania informacji (\textit{active recall}), co zmusza mózg do większego wysiłku poznawczego niż bierne czytanie, prowadząc do trwalszego śladu pamięciowego. Jednakże potencjał tej metody w środowisku cyfrowym jest ściśle uzależniony od jakości narzędzia, które ją obsługuje. Aplikacja wspierająca proces nauki nie może stanowić bariery technologicznej; musi być intuicyjna, wydajna i dostępna, aby użytkownik mógł skupić się na merytoryce, a nie na obsłudze interfejsu.

% Niniejsza praca inżynierska stanowi odpowiedź na te wyzwania, prezentując proces projektowania i implementacji nowoczesnej aplikacji internetowej do zarządzania zestawami fiszek, która łączy sprawdzone metodyki dydaktyczne z najnowszymi standardami wytwarzania oprogramowania webowego.

% % --- CZĘŚĆ 2: ANALIZA PROBLEMU (Tutaj wchodzimy w szczegóły "dlaczego to robisz") ---

% \section{Motywacja i analiza obszaru problemowego}

% Decyzja o podjęciu tematu budowy systemu do zarządzania fiszkami wynika z obserwacji rynku istniejących rozwiązań oraz analizy potrzeb współczesnych użytkowników. Mimo dostępności wielu platform edukacyjnych (takich jak Anki czy Quizlet), wciąż istnieje zauważalna luka pomiędzy zaawansowaniem algorytmicznym a użytecznością interfejsu (User Experience - UX) oraz otwartością architektury.

% Wiele popularnych narzędzi, choć skutecznych dydaktycznie, charakteryzuje się wysokim progiem wejścia. Wymagają one od użytkownika skomplikowanej konfiguracji, instalacji dedykowanego oprogramowania klient-serwer lub borykają się z długu technologicznym, co skutkuje przestarzałym i nieintuicyjnym interfejsem. Z kolei nowoczesne rozwiązania komercyjne często zamykają kluczowe funkcjonalności – takie jak praca offline, zaawansowane formatowanie tekstu czy współdzielenie materiałów – za systemami płatności (paywall).

% Kluczowym problemem zidentyfikowanym w toku analizy wstępnej jest czasochłonność procesu tworzenia materiałów. Użytkownicy często rezygnują z systematycznej nauki, ponieważ przygotowanie estetycznych i bogatych w treści fiszek zajmuje więcej czasu niż sama powtórka materiału.

% W związku z powyższym, główną motywacją realizacji niniejszej pracy stało się stworzenie rozwiązania, które eliminuje te bariery poprzez:
% \begin{enumerate}
%     \item \textbf{Obniżenie progu wejścia} – poprzez realizację aplikacji w architekturze Single Page Application (SPA), dostępnej bezpośrednio z przeglądarki bez konieczności instalacji.
%     \item \textbf{Usprawnienie procesu twórczego} – dzięki implementacji zaawansowanego edytora WYSIWYG (What You See Is What You Get) oraz integracji z modelami językowymi sztucznej inteligencji (LLM), które wspomagają kategoryzację i opis materiałów.
%     \item \textbf{Społecznościowy wymiar nauki} – umożliwienie łatwego współdzielenia zestawów, oceniania ich i komentowania, co sprzyja wymianie wiedzy w grupach studenckich czy zawodowych.
% \end{enumerate}

% % --- CZĘŚĆ 3: CEL I ZAKRES (To jest "mięso" inżynierskie z Twojego PDFa) ---

% \section{Cel i zakres pracy}

% Głównym celem pracy jest zaprojektowanie i zaimplementowanie skalowalnej aplikacji internetowej typu \textit{headless}, służącej do tworzenia, zarządzania i nauki z wykorzystaniem fiszek. System został zaprojektowany z naciskiem na wydajność, bezpieczeństwo danych oraz intuicyjność obsługi.

% \subsection{Cele szczegółowe i technologiczne}
% Realizacja celu głównego wymagała osiągnięcia szeregu celów cząstkowych o charakterze inżynierskim:
% \begin{itemize}
%     [cite_start]\item \textbf{Opracowanie architektury systemu} opartej na separacji warstwy prezentacji od warstwy logiki biznesowej (Frontend-Backend separation)[cite: 140, 141].
%     [cite_start]\item \textbf{Implementacja wydajnego API} przy użyciu frameworka FastAPI (Python), obsługującego asynchroniczne przetwarzanie żądań oraz walidację danych[cite: 214, 216].
%     [cite_start]\item \textbf{Stworzenie responsywnego interfejsu użytkownika} w technologii React z wykorzystaniem języka TypeScript, zapewniającego płynność działania zbliżoną do aplikacji natywnych (SPA)[cite: 190, 353].
%     [cite_start]\item \textbf{Zastosowanie persystencji poliglotycznej} (Polyglot Persistence), polegającej na dobraniu odpowiedniego magazynu danych do specyfiki przechowywanych informacji: relacyjnej bazy PostgreSQL dla danych strukturalnych, MinIO dla plików binarnych oraz ElasticSearch dla wyszukiwania pełnotekstowego[cite: 246, 248].
%     [cite_start]\item \textbf{Zapewnienie bezpieczeństwa} poprzez implementację mechanizmów autoryzacji opartych o tokeny JWT przechowywane w ciasteczkach HttpOnly oraz ochronę przed atakami CSRF i XSS[cite: 112, 485].
%     [cite_start]\item \textbf{Konteneryzacja środowiska} z wykorzystaniem platformy Docker, co ułatwia wdrożenie i skalowanie systemu[cite: 165, 930].
% \end{itemize}

% \subsection{Zakres funkcjonalny}
% Aplikacja oferuje użytkownikom możliwość rejestracji i zarządzania kontem, tworzenia hierarchicznej struktury folderów i zestawów fiszek, a także korzystania z trybu nauki (odwracanie, tasowanie kart). [cite_start]Istotnym elementem zakresu pracy jest moduł społecznościowy, pozwalający na upublicznianie materiałów, dodawanie komentarzy oraz ocenianie zestawów[cite: 39, 45]. [cite_start]Dodatkowo system został zintegrowany z zewnętrznym API (Google Gemini) w celu automatycznego generowania tagów dla zestawów fiszek, co usprawnia proces ich wyszukiwania[cite: 760].

% \section{Układ pracy}

% Niniejsza praca składa się z dwóch głównych rozdziałów merytorycznych oraz wstępu i zakończenia.
% \begin{itemize}
%     [cite_start]\item \textbf{Rozdział 1: Projekt Systemu} – przedstawia analizę wymagań funkcjonalnych i niefunkcjonalnych, opisuje architekturę systemu w modelu klient-serwer oraz model danych uwzględniający relacyjną bazę danych i dodatkowe magazyny danych[cite: 27, 28].
%     [cite_start]\item \textbf{Rozdział 2: Implementacja Systemu} – zawiera szczegółowy opis procesu wytwarzania oprogramowania, struktury kodu warstwy prezentacji i logiki, a także omawia kluczowe rozwiązania techniczne, takie jak hybrydowe zarządzanie stanem, telemetria (OpenTelemetry) oraz integracja z usługami zewnętrznymi[cite: 340, 341].
% \end{itemize}
% Podsumowanie pracy zawiera wnioski końcowe oraz kierunki dalszego rozwoju aplikacji.
% \section{Analiza porównawcza}
% \section{Cel i zakres pracy}






% \chapter{Wstęp}
% \label{ch:wstep}

% Zanim przejdziesz dalej musisz odpowiedzieć sobie na bardzo ważną kwestię. Czy chcesz podczas pisania pracy dyplomowej ręcznie kontrolować, sprawdzać i~wciąż na nowo ustawiać takie rzeczy jak:

% \begin{itemize}
% 	\item wielkość i~rodzaj czcionki,
% 	\item długość wcięcia akapitu,
% 	\item styl tytułów rozdziałów i~podrozdziałów,
% 	\item numeracja stron, rysunków, tabel, listingów, rozdziałów i~innych elementów,
% 	\item spis treści i~jego numery stron,
% 	\item zgodność indeksacji bibliografii z~tekstem,
% 	\item wiele innych drobiazgów, o~których możesz nawet jeszcze nie wiedzieć?
% \end{itemize}

% \textit{A może chcesz się od tego uwolnić i~po prostu napisać pracę dyplomową?}

% Jeśli preferujesz pierwsze podejście, to z~pewnością lepiej będzie, jeśli napiszesz pracę ,,jak wszyscy w~łerdzie'' lub podobnym edytorze tekstu. Jeśli jednak chcesz \textit{skupić się na treści}, to zachęcam do skorzystania z~systemu składu tekstu \LaTeX{} i~szablonu pracy dyplomowej, który właśnie oglądasz. W~obu przypadkach efekt końcowy będzie adekwatny do możliwości wybranego narzędzia.

% Co wybierasz?

% \begin{figure}[!htbp]
% 	\centering \includegraphics[width=0.618\linewidth]{notneo.png} % 0.618 to złoty podział, bardzo dobra szerokość
% 	\caption{,,Not Neo'', ChatGPT, 2025}
% 	\label{rys:notneo}
% \end{figure}

% Cieszę się, że nadal czytasz ten tekst. A~to znaczy, że wybrane przez Ciebie rozwiązanie to \LaTeX. Jednak nie ma nic za darmo. Jeśli jest to dla Ciebie nowe narzędzie, to musisz je choć trochę poznać.

% %%%%%%%%%%%%%%%%%%%%%%%%%%%%%%%%%%%%%%%%%%%%%%%%
% \section{Rysunki}
% Wydawać by się mogło, że edycja wzorów albo tabel jest największą bolączką początkujących użytkowników \LaTeX{a}. To zaskakujące ale okazuje się, że najwięcej trudności sprawiają rysunki.

% Rysunki w~\LaTeX{u} wstawiane są za pomocą polecenia \texttt{includegraphics}. Najczęściej używa się go wewnątrz tak zwanego ,,otoczenia'' \texttt{figure}. To otoczenie pozwala na dodanie podpisu (\texttt{caption}).

% Ponadto w~tymże otoczeniu można zdefiniować unikatowy znacznik (\texttt{label}), dzięki któremu można się później w~tekście do tego rysunku odwołać za pomocą \texttt{ref}. Nowo dodany znacznik, jak kilka innych rzeczy w~\LaTeX{u}, może wymagać dwukrotnego uruchomienia kompilacji, gdyż za pierwszym razem \LaTeX{} rejestruje ich obecność a~za drugim już ,,wie'', dokąd mają prowadzić odwołania.

% Przykładem jest rysunek~\ref{rys:notneo}, który być może powinien znaleźć się na stronie~\pageref{ch:wstep}, zaraz poniżej słów ,,co wybierasz''. Byłoby fajnie. Jednak tam się nie zmieścił. Dlatego \LaTeX{} przeniósł go na następną stronę. \textbf{To typowe i~normalne zachowanie, z~którym nie należy walczyć.} Dla wielu początkujących użytkowników jest to największa mentalna bariera, którą muszą pokonać. Wynika ona ze złych doświadczeń z~edytorami tekstu, gdzie każdy rysunek trzeba ręcznie ustawić i~zadbać, żeby na pewno wpasował się w~treść na odpowiedniej stronie. A~on i~tak później się przesunie\ldots

% \textbf{Dlatego trzeba oduczyć się konieczności ciągłego kontrolowania tak zwanego formatowania.} Celem jest przecież \textit{napisanie} pracy dyplomowej.

% Zdarza się, że w~trakcie pisania pracy dyplomowej rysunki nieoczekiwanie ,,fruwają'' gdzieś w~tekście. Czasem \LaTeX{} tworzy z~nich wielką ,,galerię obrazków'' na końcu rozdziału. Nie należy się temu dziwić. Skoro autor tekstu miał intencję zdominować pracę rysunkami, to taki właśnie dokument powstanie. Aby sobie z~tymi problemami radzić, przede wszystkim należy możliwie długo wstrzymać się z~ingerencją w~położenie grafik. Najlepiej zostawić to na sam koniec pisania pracy, gdyż wówczas już ma sens zastosowanie działań takich, jak opisane niżej.

% Być może oczekiwany efekt przeniesie przesunięcie otoczenia \texttt{figure} trochę wyżej lub niżej w~źródle \LaTeX{}? Czasem aby odniosło to pożądany skutek należy przenieść je nawet o~kilka akapitów. Trzeba tylko pamiętać o~pustej linii przed i~po nim w~pliku źródłowym.

% Można spróbować zmiany sugestii umieszczenia grafiki, czyli h -- \textit{here}, t -- \textit{top}, b -- \textit{bottom}, oraz ewentualnie skorzystanie z~wykrzyknika, który stanowi ,,gorącą prośbę do \LaTeX{a}, żeby się bardziej postarał''.

% Skoro rysunki są za duże to może należy je zmniejszyć? Nie każdy rysunek musi zajmować 2/3 szerokości strony. Warto to rozważyć, o~ile zostanie zachowana czytelność ich zawartości. Stopień pisma (wielkość czcionki) wewnątrz rysunku nie może być mniejszy niż wielkość czcionki podpisu pod rysunkiem. Jest to bardzo ważne w~odniesieniu do wykresów i~schematów. Dla treningu możesz spróbować zmniejszyć szerokość rysunku~\ref{rys:notneo} z~0.618 do 0.3 (\texttt{\textbackslash linewidth}) i~zobaczyć, jak wynikowy PDF będzie wyglądał po rekompilacji.

% Być może to nie rysunek jest za duży a~treści jest za mało? Do każdego rysunku trzeba odwołać się w~tekście pracy za pomocą \texttt{ref}, tak jak powyżej i~tu ponownie jest odwołanie do rysunku~\ref{rys:notneo}. Zauważ, że \LaTeX{} dba o~spójność numeracji. Każdy rysunek musi być przywołany oraz opisany. Ilość opisującej go treści powinna być porównywalna z~jego powierzchnią. Jeśli nie można wytworzyć takiego opisu a~przecież ,,obraz jest wart tysiąca słów'', to być może ten rysunek niczego do pracy nie wnosi i~po prostu należy go usunąć?

% Pozycjonując elementy graficzne a~rysunki w~szczególności, należy zapomnieć o~\underline{złych} poradach w~stylu ,,rysunek powinien pojawić się przed odwołaniem do niego'' albo ,,rysunek powinien być na tej samej stronie, co odwołanie do niego''. Zamiast tego należy wdrożyć swobodniejsze podejście: rysunek powinien być \textbf{możliwie blisko odwołania do niego ale nie bliżej}. Wystarczy, jeśli po ewentualnym wydrukowaniu tekstu czytelnik nie będzie zmuszany do przewracania kartek w~celu znalezienia rysunku, o~którym czyta.

% W~\LaTeX{u} praktycznie nie ma ryzyka, że wszystko ,,się rozjedzie'', gdy przesuniemy rysunek. Ewentualne zmiany są też w~pełni odwracalne i~znacznie łatwiejsze, niż w~typowych edytorach tekstu.

% Natomiast zdecydowanie \textbf{nie wolno} używać sztuczek, które niszczą sens zautomatyzowanego składu tekstu, czyli takich jak na przykład:
% \begin{itemize}
% 	\item FloatBarrier
% 	\item pagebreak
% 	\item newpage
% 	\item clearpage
% \end{itemize}

% Powyższe zasady oraz porady można zastosować wobec pozostałych elementów graficznych takich jak tabele i~listingi.

% Rysunki rastrowe powinny mieć minimalną rozdzielczość 300 dpi (punktów na cal). Dlatego rysunek zajmujący szerokość strony powinien mieć minimum 1900 pikseli szerokości. Dobry wydruk rastrowy to przynajmniej 600 dpi. Można uzyskać ,,nieskończoną'' rozdzielczość stosując grafikę wektorową. Możliwe jest wyeksportowanie grafik wektorowych do formatu PDF i~użycie ich w~\texttt{includegraphics}.

% Inne ,,technikalia'' (\TeX{nikalia}?) wykorzystania \LaTeX{a} są krótko omówione i~zaprezentowane w~dodatku~\ref{app:latex} niniejszego szablonu.

% \section{To wspaniale, ale co z~tym wstępem?}

% Wstęp pracy dyplomowej powinien zawierać wprowadzenie do zagadnienia na tyle ogólne, żeby również osoba nie mająca wiedzy w~przedstawianym temacie była w~stanie zrozumieć, czego ta praca dotyczy. Opisać należy to, co ,,jest'', a~więc używając czasu teraźniejszego.

% Warto we wstępie pokazać zagadnienie za pomocą grafiki (rysunek, schemat, tabela, wykres), aby ułatwić czytelnikowi szybkie rozpoznanie, czego dotyczy dana praca.

% Warto posłużyć się odnośnikami bibliograficznymi do źródeł o~niskim progu wymagań -- na przykład czasopism popularnonaukowych lub branżowych.

% \textbf{Niezbędnym elementem wstępu jest przedstawienie celu pracy.} Najprościej zacząć zdanie od słów ,,celem pracy jest''. Ponadto można przedstawić cele szczegółowe, założenia pracy, zaplanowane etapy realizacji badań i~innych działań, które będą podjęte w~ramach realizacji pracy dyplomowej dla osiągnięcia jej głównego celu.

% Określając powyższe aspekty pracy warto pamiętać, że w~ramach pracy inżynierskiej można podejmować działania
% projektowe, technologiczne, organizacyjne i~badawcze. Praca inżynierska powinna przedstawiać weryfikację poprawności uzyskanego rezultatu. Natomiast praca magisterska \textbf{musi} dodatkowo zawierać ,,element badawczy'' a~więc pod wieloma względami powinna stać na wyższym poziomie niż praca inżynierska.

% Jeśli omówienie celu pracy i~elementów pokrewnych jest dość długie to może stanowić podrozdział ,,Wstępu''. Aczkolwiek ,,Wstęp'' jako całość raczej nie powinien być tak długi, by wymagał większej liczby podrozdziałów. Teoretyczne podstawy pracy powinny zostać dokładniej przedstawione w~kolejnym rozdziale.

    \chapter{Projekt Systemu}
\label{ch:projekt_systemu}

Przed rozpoczęciem implementacji systemu, kluczowym etapem było zdefiniowanie jego projektu. 
Faza ta obejmowała określenie zbioru wymagań zarówno funkcjonalnych, jak i niefunkcjonalnych na 
podstawie wcześniej przeprowadzonej analizy porównawczej. Zdefiniowane wymagania były fundamentem 
projektu architektury ułatwiającym późniejszą pracę nad implementacją aplikacji.

\section{Wymagania funkcjonalne}

Wymagania funkcjonalne określają funkcje i zachowania, które musi spełniać system, aby w pełni zrealizować określone wymagania użytkowników \cite{sommerville2015}. 
W ramach stworzenia wymagań funkcjonalnych zdefiniowany został szczegółowy zakres działania aplikacji oraz 
zidentyfikowane zostały wszystkie niezbędne funkcje systemu. 

Początkowa lista sprecyzowanych wymagań funkcjonalnych szybko rozrosła się do rozmiarów ciężkich w zarządzaniu. 
Z tego powodu podjęto decyzje o pogrupowaniu powiązanych ze sobą funkcji w siedem głównych modułów. Taki podział 
pozwolił na sensowną separację głównych funkcjonalności systemu, co ułatwiło późniejsze modelowanie architektury.

Poniższy diagram (Rys \ref{rys:use_case_diagram}) ilustruje powyżej wspomniany podział na siedem wysokopoziomowych modułów:

\begin{itemize}
    \item \textbf{Przeglądanie i nauka fiszek}: Komponent umożliwiający odwracanie, tasowanie i przeglądanie fiszek.
    \item \textbf{Przeglądanie i wyszukiwanie publicznych materiałów}: Moduł dostępny dla niezalogowanych użytkowników pozwalający na przeszukiwanie publicznych zasobów.
    \item \textbf{Tworzenie i edycja fiszek}: Zaawansowany moduł do tworzenia fiszek w tym edytor WYSIWYG umożliwiający formatowanie tekstu i wstawianie obrazków.
    \item \textbf{Uwierzytelnianie i zarządzanie kontem}: Obejmuje procesy rejestracji, logowania i wylogowywania oraz zarządzanie sesją użytkownika.
    \item \textbf{Zarządzanie materiałami}: Pozwala na tworzenie, przenoszenie i edytowanie nazw materiałów.
    \item \textbf{Dodawanie komentarzy i ocen}: Moduł umożliwiający interakcję między użytkownikami poprzez komentowanie i ocenianie fiszek, i komentarzy.
    \item \textbf{Udostępnianie oraz kopiowanie materiałów}: Komponent umożliwiający udostępnianie własnych zestawów innym użytkownikom i zarządzanie ich uprawnieniami.
\end{itemize}

\begin{figure}[!htbp]
	\centering \includegraphics[width=0.75\linewidth]{Use_case_diagram_thesis.pdf} % 0.618 to złoty podział, bardzo dobra szerokość
	\caption{Wysokopoziomowy diagram modułów}
	\label{rys:use_case_diagram}
\end{figure}

\subsection*{Uwierzytelnianie i zarządzanie kontem}
Moduł Uwierzytelniania i zarządzania kontem definiuje wszystkie operacje związane z uwierzytelnianiem użytkowników i kontrolą dostępu.

\subsubsection*{Wymagania funkcjonalne:}
\begin{itemize}
    \item Użytkownik może zarejestrować się za pomocą adresu e-mail i hasła.
    \item Użytkownik może zalogować się za pomocą adresu e-mail i hasla.
    \item Użytkownik może wylogować się z aplikacji, czyszcząc przy tym sesję.
    \item System zapisuje sesję zalogowanego użytkownika po udanym logowaniu.
\end{itemize}

\subsection*{Zarządzanie materiałami}
Moduł ten opisuje wszystkie operacje, jakie mogą być wykonywane w eksploratorze plików aplikacji.
Elementem przechowywanym w eksploratorze plików jest materiał reprezentujący zestawy fiszek i foldery.

\subsubsection*{Wymagania funkcjonalne:}
\begin{itemize}
    \item Użytkownik może tworzyć, zmieniać nazwę i usuwać materiały.
    \item Użytkownik może zagnieżdżać foldery tworząc hierarchiczną strukturę.
    \item Użytkownik może przenosić materiały między folderami.
    \item Użytkownik może wyszukiwać materiały w obrębie danego folderu po ich nazwie.
\end{itemize}

\subsection*{Tworzenie i edycja zestawów fiszek}
Komponent ten odpowiada za tworzenie i modyfikację zestawu fiszek. Każdy zestaw 
oprócz pojedyńczych fiszek zawiera także metadane określające jego atrybuty (nazwa, opis i dostępność). 

\subsubsection*{Wymagania funkcjonalne:}
\begin{itemize}
    \item Użytkownik może tworzyć i usuwać zestawy fiszek.
    \item Użytkownik może edytować atrybuty zestawu fiszek (nazwa, opis, dostępność).
    \item Użytkownik może dodawać, usuwać i edytować pojedyncze fiszki w ramach danego zestawu fiszek.
    \item Użytkownik może formatować styl tekstu zawartego na fiszce poprzez np. kursywę czy pogrubienie.
\end{itemize}

\subsection*{Przeglądanie i nauka fiszek}
Moduł przeglądania określa funkcje systemu, jakie można wykorzystywać podczas nauki fiszek.

\subsubsection*{Wymagania funkcjonalne:}
\begin{itemize}
    \item Użytkownik może przeglądać dowolną fiszkę z wybranego zestawu fiszek, do którego posiada dostęp.
    \item Użytkownik może odwrócić fiszkę, żeby zobaczyć jej rewers.
    \item Użytkownik może potasować fiszki w zestawie, losowo zmieniając ich kolejność.
    \item Użytkownik może zamienić stronę fiszek (awers/rewers), w jakiej podstawowo się wyświetlają.
\end{itemize}

\subsection*{Przeglądanie i wyszukiwanie publicznych materiałów}
W tym module opisane są operacje dostępne dla wszystkich użytkowników. Są to głównie funkcje systemu opierające się na
publicznych zestawach, takie jak wyszukiwanie lub przeglądanie.

\subsubsection*{Wymagania funkcjonalne:}
\begin{itemize}
    \item System umożliwia wyszukiwanie publicznych zestawów fiszek na podstawie słów kluczowych.
    \item System prezentuje publiczne zestawy pogrupowane w kategorie (najpopularniejsze, najbardziej lubiane, najnowsze).
    \item Użytkownik może filtrować kategorie publicznych zestawów według określonych ram czasowych, np. najpopularniejsze w ostatnim tygodniu. 
\end{itemize}

\subsection*{Udostępnianie, współdzielenie oraz kopiowanie materiałów}
Moduł ten definiuje funkcje współpracy między użytkownikami.
Użytkownicy mogą udostępniać swoje prywatne zestawy fiszek innym i zarządzać uprawnieniami innych użytkowników.

\subsubsection*{Wymagania funkcjonalne:}
\begin{itemize}
    \item Użytkownik może udostępnić swój prywatny zestaw fiszek innemu zarejestrowanemu użytkownikowi. 
    \item Użytkownik może zarządzać uprawnieniami użytkowników do udostępnionego materiału (przeglądający/edytujący).
    \item Użytkownik może zaakceptować lub odrzucić zaproszenie do udostępnionego materiału.
    \item Użytkownik może stworzyć kopię dowolnego zestawu, do którego ma dostęp.
\end{itemize}

\subsection*{Dodawanie komentarzy i ocen}
Komponent ten realizuje funkcje społecznościowe aplikacji. Pozwalają one na interakcję
między użytkownikami i ocenę komentarzy/zestawów fiszek.

\subsubsection*{Wymagania funkcjonalne:}
\begin{itemize}
    \item Użytkownik może dodawać, edytować i usuwać własne komentarze pod zestawami fiszek. 
    \item System zezwala na zagnieżdżanie komenatrzy do najwyżej jednego poziomu.
    \item Użytkownik może oceniać fiszki i komentarze za pomocą zostawiania łapki w górę/dół.
\end{itemize}

\section{Wymagania niefunkcjonalne}
\label{sec:wymagania_niefunkcjonalne}

Dzięki wymaganiom funkcjonalnym zostało już zdefiniowane, co aplikacja ma robić.
Równie istotne jest określenie, jak ma to robić. Wymagania niefunkcjonalne definiują standardy i cechy, 
jakie system musi spełniać, aby działać skutecznie i wydajnie \cite{sommerville2015}. Sprecyzowanie poniższych wymagań niefunkcjonalnych było
podstawą późniejszego modelowania architektury systemu.

\subsection*{Bezpieczeństwo}
System musi zapewniać ochronę danych użytkowników oraz integralność całej aplikacji.

\begin{itemize}
    \item \textbf{Zarządzanie sesją:} System musi implementować bezpieczny sposób zarządzania sesją minimalizujący ryzyko jej przechwycenia.
    \item \textbf{Autoryzacja:} System musi kontrolować dostęp użytkowników do zasobów. Użytkownik niebędący włascicielem materiału lub niemający do niego uprawnień nie powinien uzyskać do niego dostępu.
    \item \textbf{Sanityzacja:} System musi sanityzować wszystkie dane wejściowe, aby uniemożliwić wstrzyknięcie złośliwego oprogramowania.
    \item \textbf{Przechowywanie haseł:} Dane poufne użytkowników takie jak np. hasła muszą być zahaszowane przy użyciu silnego algorytmu kryptograficznego przed zapisaniem do bazy danych.
    \item \textbf{Walidacja plików:} System musi walidować wszystkie pliki przesyłane przez użytkowników pod kątem rozmiaru, typu i możliwie zagnieżdzonego złośliwego kodu.
\end{itemize}

\subsection*{Wydajność}
System musi zapewniać szybki czas odpowiedzi oraz płynne działanie.

\begin{itemize}
    \item \textbf{Czas odpowiedzi wyszukiwania:} Czas odpowiedzi wyszukiwania publicznych materiałów powinien średnio zajmować mniej niż dwie sekundy.
    \item \textbf{Operacje asynchroniczne:} Czasochłonne operacje wejścia/wyjścia (I/O) nie mogą blokować interfejsu użytkownika ani serwera.
    \item \textbf{Buforowanie danych:} Aplikacja po stronie klienta musi buforować część danych pobieranych z serwera w celu zredukowania liczby zapytań i niepotrzebnego obciążania systemu.
    \item \textbf{Wydajność zapytań:} System musi efektywnie pobierać złożone hierarchiczne struktury dancyh, aby nie powodować nadmiernego obciążania bazy danych.
\end{itemize}

\subsection*{Niezawodność i Odporność}
System musi być stabilny i odporny na błędy.

\begin{itemize}
    \item \textbf{Monitorowanie i logowanie:} System musi zapewniać monitorowanie stanu aplikacji, wydajności serwera oraz logowanie błędów.
    \item \textbf{Integralność danych:} System musi gwarantować spójność danych. Usunięcie zasobu nadrzędnego musi powodować usunięcie wszystkich zasobów podrzędnych. 
\end{itemize}

\subsection*{Użyteczność}
Interfejs użytkownika musi być intuicyjny i łatwy w użyciu
\begin{itemize}
    \item \textbf{Edytor fiszek:} System musi zapewniać intuicyjny i łatwy w użyciu edytor tekstu (WYSIWYG) do tworzenia fiszek umożliwiajacy formatowanie tekstu i dodawanie obrazków.
    \item \textbf{Ochrona przed błędami użytkownika:} System musi chronić użytkowników przed przypadkową utratą danych poprzez wymóg potwierdzenia przed każdą próbą usunięcia zasobu.
    \item \textbf{Przepływ pracy:} Aplikacja wspiera płynny przepływ pracy użytkownika poprzez automatyczne przekierowania np. po udanym zalogowaniu.
\end{itemize}

\subsection*{Skalowalność i łatwość utrzymania}
System musi być przygotowany na wzrost liczby aktywnych użytkowników lub gromadzonych danych. 

\begin{itemize}
    \item \textbf{Skalowalność horyzontalna:} Serwer musi być bezstanowy (stateless), aby umożliwić uruchomienie jego wielu instancji i rozdzielenie ruchu sieciowego (load balancing).
    \item \textbf{Separacja komponentów:} Kluczowe komponenty (np. logika aplikacji, system wyszukiwania.) powinny być od siebie oddzielone, aby umożliwić ich niezależne skalowanie w przyszłości.
    \item \textbf{Wersjonowanie schematu bazy danych:} Zmiany w schemacie bazy danych muszą być zarządzane za pomocą migracji, pozwalających na śledzenie historii zmian.
\end{itemize}

\section{Architektura systemu}

System został zaprojektowany według standardowego wzorca architektonicznego klient-serwer w modelu headless \cite{fowler2002}. 
Oznacza to całkowitą separację aplikacji klienckiej ('głowy') od serwera ('ciała'), który jedynie udostępnia swoje API. 
Podstawą tej separacji jest logiczny podział systemu na trzy warstwy tworzące architekturę trójwarstwową (3-Tier Architecture) \cite{sommerville2015}:
\begin{itemize}
    \item \textbf{Warstwa Prezentacji} -- Jest to warstwa interfejsu użytkownika. Zajmuje się wyświetlaniem i zbieraniem danych wprowadzonych przez użytkownika.
    \item \textbf{Warstwa Logiki} -- Jest odpowiedzialna za przetwarzanie wszystkich żądań od użytkownika. Jest pośrednikiem między warstwą prezentacji a danych, kordynując przepływ danych między nimi.
    \item \textbf{Warstwa Danych} -- Warstwa odpowiedzialna za zarządzanie i przechowywanie danych. 
\end{itemize}
Taka separacja odpowiedzialności (separation of concerns) ułatwia zarządzanie kodem i przyszłe skalowanie aplikacji \cite{cleancode}.

Aplikacja kliencka została zrealizowana jako Single Page Application (SPA).
Całe przetwarzanie i zarządzanie danymi odbywa się na serwerze (w warstwach logiki i danych), 
a zadaniem klienta (warstwy prezentacji) jest renderowanie i obsługa interfejsu użytkownika oraz wysyłanie żądań do serwera.
Komunikacja między klientem a serwerem odbywa się asynchroniczne przy wykorzystaniu protokołu HTTP \cite{tanenbaum_networks}. Cały przesył danych jest realizowany 
przy pomocy obiektów transferów danych (Data Transfer Objects) \cite{fowler2002}, które przed wysyłką są serializowane do formatu JSON.

Warto zauważyć, że poszczególne komponenty aplikacji (klient, serwer itp.) zostały skonteneryzowane 
przy użyciu platformy Docker \cite{docker_docs}, co pozwala na większe rozdzielenie odpowiedzialności i ułatwia skalowanie aplikacji
w przyszłości. 
Cała architektura została zamodelowana na diagramie (Rys. \ref{rys:architecture_diagram}) przedstawiającym
poszczególne komponenty i interakcje między nimi.

\begin{figure}
    \centering \includegraphics[width=0.85\linewidth]{architecture_diagram_wieksze_czcionki_v2.pdf} % 0.618 to złoty podział, bardzo dobra szerokość
	\caption{Diagram architektury systemu}
	\label{rys:architecture_diagram}
\end{figure}


\subsection{Warstwa prezentacji}
Pierwszym etapem projektowania warstwy prezentacji było zamodelowanie najistotniejszych ekranów 
interfejsu użytkownika za pomocą narzędzia Figma. Stworzenie makiet kluczowych ekranów pozwoliło na 
wydzielenie reużywalnych komponentów, zdefiniowanie spójnych stylów i struktury projektu dla całej warstwy, 
co znacząco przyśpieszyło dalszą fazę projektowania i implementacji.

\begin{figure}
    \centering \includegraphics[width=0.85\linewidth]{figma_materials_screen.pdf} % 0.618 to złoty podział, bardzo dobra szerokość
	\caption{Ekran materiałów zamodelowany w figmie}
	\label{rys:figma_materials_screen}
\end{figure}

Interfejs użytkownika został zaimplementowany jako aplikacja typu Single Page Application (SPA). SPA to aplikacja 
składająca się technicznie tylko z jednej strony (jednego pliku HTML), a cała nawigacja i zmiany widoków odbywają się 
dynamicznie poprzez modyfikację drzewa DOM (Document Object Model) po stronie przeglądarki użytkownika \cite{react_docs}. 
Główną rolą zaprojektowanej aplikacji klienckiej jest wizualizacja danych otrzymanych od serwera oraz obsłużenie jej interkacji z użytkownikiem.    

\begin{figure}
    \centering \includegraphics[width=0.85\linewidth]{figma_flashcard_screen.pdf} % 0.618 to złoty podział, bardzo dobra szerokość
	\caption{Ekran przeglądania zestawu fiszek zamodelowany w figmie}
	\label{rys:figma_flashcard_screen}
\end{figure}

Do implementacji aplikacji użyto biblioteki React \cite{react_docs} wraz z TypeScriptem, który jest nadzbiorem języka JavaScript.
Taki stos technologiczny pozwolił na pisanie czystszego, bardziej skalowalnego i odpornego na błędy kodu. Do obsłużenia 
nawigacji po stronie klienta posłużyła biblioteka React Router, dzięki której nie ma konieczności przeładowywania 
całej strony przy zmianie widoku, co pozwala na realizację architektury SPA. 

% Bardzo ważnym elementem w projektowaniu interfejsu użytkownika jest koncept zarządzania stanem. W tym przypadku zdecydowano się 
% na podejście hybrydowe. Globalny stan aplikacji, taki jak np. status uwierzytelnienia lub prywatne materiały użytkownika, jest 
% obsługiwany za pomocą Redux Toolkit co zwiększa wydajność w przypadku długoterminowych danych, które utrzymują się aż do odświeżenia strony. 
% Aby spełnić wymagania niefunkcjonalne, konieczne było zastosowanie biblioteki React Query, która zezwala na tymczasowe buforowanie danych.
Bardzo ważnym elementem w projektowaniu interfejsu użytkownika jest koncept zarządzania stanem. W tym przypadku zdecydowano się 
na podejście hybrydowe:
\begin{itemize}
    \item \textbf{Globalny stan} taki jak np. status uwierzytelnienia czy prywatne materiały użytkownika, jest 
obsługiwany za pomocą Redux Toolkit.
    \item \textbf{Dane tymczasowe} takie jak np. najbardziej popularne czy najczęściej wyświetlane zestawy fiszek na stronie publicznej
są tymczasowo buforowane dzięki bibliotece React Query.
\end{itemize}
Takie podejście hybrydowe zapewniło scentralizowanie stanu dla długoterminowych niezmiennych danych i łatwe zarządzanie danymi 
tymczasowymi poprzez czasowe buforowanie. 

Najważniejszym elementem interfejsu użytkownika realizującym jedno z kluczowych wymagań funkcjonalnych dotyczących tworzenia fiszek jest edytor WYSIWYG. Istnieje
wiele gotowych  rozwiązań w internecie, ale niewiele z nich oferowało elastyczność wymaganą do zaimplementowania wszystkich niezbędnych funkcjonalności. 
Do implementacji takiego edytora wybrano framework TipTap. TipTap to edytor typu 'headless' czyli nie dostarcza żadnego gotowego interfejsu a jedynie samą logikę obsługi edytora.
Dzięki temu możliwe było stworzenie niestandardowych rozszerzeń realizujących bardziej skomplikowane operacje. Przykładem takiego niestandardowego rozszerzenia może być np. komponent 'ImageGrid', pozwalający na 
wstawianie wielu obrazków w jednym rzędzie.

\subsection{Warstwa logiki}
Warstwa logiki pełni rolę serwera aplikacji udostępniającego interfejs REST API \cite{fielding_rest}. Jest ona odpowiedzialna za całą logikę biznesową, 
przetwarzanie i bezpieczeństwo danych. Do implementacji serwera wykorzystano framework FastAPI \cite{fastapi_docs}. FastAPI 
to nowoczesny framework webowy dla języka python pozwalający na tworzenie wydajnych API w możliwie najkrótszym 
czasie. FastAPI jest zbudowany na podstawie standardu ASGI (Asynchronous Server Gateway Interface), co pozwala na 
współbieżne przetwarzanie zapytań klienta i jednoczesną obsługę wielu operacji wejścia/wyjścia (I/O). Dzięki integracji 
z biblioteką Pydantic FastAPI automatycznie waliduje wszystkie dane przychodzące z żądań HTTP, a także serializuje dane zwracane do klienta.

Backend został zaimplementowany w oparciu o model wielowarstwowy. Takie podejście pozwala na wyraźne rozdzielenie 
odpowiedzialności, co czyni kod czystszym i łatwiejszym w zarządzaniu. Wyróżnione zostały 3 główne warstwy:
\begin{itemize}
    \item \textbf{Tras (Routes)} -- Odpowiada za definiowanie punktów końcowych API (endpointów), zarządzanie 
    żądaniami HTTP oraz wysyłanie odpowiedzi. Parsuje także dane wejściowe za pomocą modeli pydantic i przekazuje je do 
    odpowiednich serwisów. 
    \item \textbf{Usług (Services)} -- W tej warstwie znajdują się cała logika aplikacji. Działa ona jako pośrednik 
    między warstwą tras a repozytoriów, przetwarzając i przekazując dane uzyskane od warstwy tras do repozytoriów. 
    \item \textbf{Repozytoriów (Repositories)} -- Odpowiada za dostęp do danych. Wszystkie dane są przez nią zarządzane 
    pod kątem zapisu i odczytu z warstwy danych.
\end{itemize}
\begin{figure}[h!]
    \centering \includegraphics[width=1\linewidth]{backend.pdf} % 0.618 to złoty podział, bardzo dobra szerokość
	\caption{Diagram architektury warstwy logiki}
	\label{rys:backend}
\end{figure}
Centralnym mechanizmem łączącym te warstwy jest wzorzec wstrzykiwania zależności (dependency injection). Wzorzec ten, zapewniający luźne powiązania między obiektami (loose coupling), został wykorzystany  
do przekazywania instancji serwisów do poszczególnych tras oraz do zarządzania cyklem życia zasobów \cite{seemann_di}. Dzięki temu połączenie do bazy danych 
jest automatycznie tworzone dla przychodzącego zapytania i zamykane po wysłaniu odpowiedzi do klienta. 

Serwer został zaprojektowany jako bezstanowy, co oznacza, że sesja użytkownika jest przechowywana po stronie przeglądarki użytkownika, 
a nie po stronie serwera. Jest to potrzebne, aby spełnić kluczowe założenie REST API, które wymaga, żeby w każdym żądaniu zostały przesłane 
wszystkie informacje potrzebne do jego obsłużenia. Taka architektura pozwala na skalowanie horyzontalne, gdzie każda instancja serwera może 
obsługiwać dowolnego użytkownika. Aplikacja, aby spełnić to założenie, używa tokenów JWT (JSON Web Tokens) do uwierzytelniania użytkowników \cite{rfc7519}, które 
generowane są na serwerze przy użyciu biblioteki python-jose. Równie istotne z poziomu bezpieczeństwa aplikacji jest sanityzacja danych i zabezpieczenie haseł. 
Wszystkie dane wejściowe są sanityzowane za pomocą bibliotek nh3 (dane tekstowe) oraz pillow (obrazy), a hasła haszowane 
za pomocą biblioteki passlib. Proces haszowania polega na dodaniu soli i pieprzu do hasła, a następnie zahaszowaniu go przy pomocy algorytmu argon2, co jest zgodne 
z obecnymi zaleceniami dotyczącymi bezpieczeństwa aplikacji \cite{owasp_passwords}.

Serwer, aby spełnić kluczowe wymaganie jakościowe dotyczące wysokiej repsonsywności systemu, musi obsługiwać długie operacje wejścia/wyjścia bez blokowania 
głównego wątku serwera i aplikacji klienta. W przypadku długotrwałych operacji takich jak np. generowanie tagów przez zewnętrzne API (Gemini) system 
wykorzystuje mechanizm zadań w tle (background task). Dzięki temu klient od razu otrzymuje odpowiedź o udanym stworzeniu zestawu fiszek, a długotrwała 
operacja generowania tagów odbywa się asynchroniczne w tle na serwerze.
\subsection{Warstwa danych}

Warstwa danych jest fundamentem każdej aplikacji, jej jedynym zadaniem jest optymalne zarządzanie i przechowywanie danych.
Ze względu na różnorodność przechowywanych danych i wymagania niefunkcjonalne dotyczące 
wydajności, system został zaprojektowany według modelu persystencji poliglotycznej (polyglot presistence) \cite{nosql_distilled}. 
Model ten polega na używaniu wielu technologii przechowywania danych, w celu zwiększenia wydajności aplikacji
poprzez wykorzystywanie rozwiązań zoptymalizowanych pod konkretne zadania. W ramach systemu wyróżniono 3 główne 
magazyny danych, z których każdy odgrywa inną rolę:
\begin{itemize}
    \item \textbf{Relacyjna baza danych (PostgreSQL)} -- Służy do przechowywania danych relacyjnych takich jak np. użytkownicy, 
    fiszki czy materiały.
    \item \textbf{Magazyn obiektów (MinIO)} -- Przechowuje pliki (dane binarne) nieposiadające żadnej struktury, takie jak np. obrazki.
    \item \textbf{Wyszukiwarkę pełnotekstową (ElasticSearch)} -- Przechowuje dane w formie dokumentów JSON, z których budowane są odwrócone
     indeksy. Indeksy te pozwalają na szybkie pełnotekstowe wyszukiwanie i ekspresową agregację danych.
\end{itemize}

\subsection*{Relacyjna baza danych}

PostgreSQL jest głównym magazynem danych aplikacji \cite{postgresql_docs}. Został on wybrany ze względu na jego niezawodność, 
wsparcie dla różnych typów danych i rozszerzeń. Przechowywane są w nim wszystkie dane użytkowników takie jak:
dane logowania, materiały, zestawy fiszek itp. Schemat bazy danych jest zarządzany przy użyciu narzędzia Liquibase, 
co pozwala na łatwe śledzenie i zarządzanie zmian.
Schemat bazy danych został zamodelowany na diagramie związków encji (ERD) (Rys. \ref{rys:database_erd_diagram}), 
a cała architektura relacyjnej bazy danych opiera się na trzech kluczowych konceptach:

\begin{figure}[h!]
    \centering \includegraphics[width=1\linewidth]{database_erd_diagram.png} % 0.618 to złoty podział, bardzo dobra szerokość
	\caption{Diagram związków encji bazy danych}
	\label{rys:database_erd_diagram}
\end{figure}

\subsubsection*{Materials}
Punktem centralnym całej bazy danych jest tabela materials. Jej zadaniem jest przechowywanie podstawowych informacji o wszystkich 
materiałach (nazwa, czas kreacji, folder, w jakim się znajduje i id właściciela). Tabela ta
pełni rolę eksploratora plików, ponieważ pozwala na przechowywanie dowolnego zasobu, dzięki wykorzystaniu
wzorca Single Table Inheritance (STI). W tabeli materials można przechowywać zarówno foldery, jak i zestawy fiszek, 
a w przypadku dalszego rozwoju aplikacji nowe zasoby takie jak np. quizy. Wszystko to jest możliwe dzięki kolumnie 
item\_type. Item\_type określa typ zasobu, jaki znajduje się w określonym wierszu (zestaw fiszek lub folder). Encja materials 
przechowuje podstawowe dane takiego zasobu, o których mowa była wcześniej, a dodatkowe informacje 
takie jak np. opis czy dostępność zestawu fiszek, znajdują się w oddzielnej tabeli połączonej z materials za pomocą klucza obcego
będącego jednocześnie kluczem głównym. Dzięki temu połączeniu encja materials staje się tabelą nadrzędną, z której dziedziczą 
tabele podrzędne takie jak np. flashcard\_sets czy w przyszłości quizes. W tabeli materials znajdują się także dwa autoreferencyjne 
klucze obce (self-referencing foreign keys):
\begin{itemize}
    \item \textbf{linked\_material\_id} -- Odpowiada za skrót do udostępnionego materiału, przez co zachowane zostaje jedno źródło prawdy, a usunięcie linku
    wskazującego na inny zestaw nie powoduje usunięcia oryginalnego zestawu.
    \item \textbf{parent\_id} -- Pozwala na tworzenie hierarchicznej struktury materiałów, gdzie każdy materiał ma przypisanego rodzica.
\end{itemize}


\subsubsection*{Comments i Votes}
Tabela comments oprócz standardowej referencji do rodzica (parent\_comment\_id) wykorzystuje także rozszerzenie PostgreSQL o nazwie ltree \cite{postgresql_docs}. 
Kolumna path wykorzystuje to rozszerzenie w celu przechowania całej ścieżki drzewa dla każdego komentarza. Pozwala to na 
pobranie całego drzewa komentarzy w jednym wydajnym zapytaniu unikając powolnego zaptania rekurencyjnego. 

Encja votes tworzy relację polimorficzną przy pomocy kolumn: votable\_id oraz votable\_type. Dzięki temu votes może zostać użyte 
do przechowywania łapek w górę/dół dla komentarzy i wszystkich materiałow (zestawów fiszek i w przyszłości quizów).


\subsubsection*{Material\_shares}
Encja material\_shares jest ściśle połączona z wcześniej opisanym linked\_material\_id. Material\_shares jest kluczową tabelą pozwalającą 
na przechowywanie i zarządzanie udostępnieniami. Jest to tabela tworząca relację wiele do wielu pomiędzy tabelami users i materials.
Nie pełni ona jednak roli prostej tabeli asocjacyjnej, ponieważ przechowuje także informację o statusie udostępnienia i nadanych uprawnieniach. 
W przypadku zaakceptowania udostępnienia, jego status w material\_shares zostanie zmieniony na zaakceptowany, a w encji materials pojawi się nowy materiał 
posiadający atrybuty oryginalnego materiału i linked\_material\_id wskazujący na oryginalny materiał.

\subsection*{Magazyn obiektów}
Przechowywanie dużych plików binarnych (BLOB) w relacyjnej bazie danych jest mało efektywne i niewydajne. Do rozwiązania tego problemu 
zastosowano system magazynowania obiektów MinIO. Serwer, aby przechować obraz, przesyła go do MinIO (który działa jako oddzielny serwis), 
a ten zapisuje obraz w dedykowanym kubełku (bucket). Link do każdego obrazka jest zapisywany w relacyjnej bazie danych, bezpośrednio 
w tekście fiszki. Kubełek ma ustawione pubiliczne uprawnienia odczytu więc klient nie potrzebuje żadnych dodatkowych uprawnień do 
wyświetlenia obrazka co znacznie upraszcza implementacje.

\subsection*{Wyszukiwanie pełnotekstowe}
System, aby spełnić wymaganie jakościowe dotyczące czasu odpowiedzi wyszukiwania, implementuje serwis wyszukiwania pełnotekstowego \cite{elasticsearch_guide}. 
ElasticSearch został użyty jako silnik wyszukiwania pełnotekstowego, gdyż umożliwia on zarówno przechowywanie danych, jak i tworzenie z nich indeksów odwróconych. 
Indeks odwrócony mapuje terminy na listę dokumentów je zawierających, co pozwala na natychmiastowe wyszukiwanie wszystkich dokumentów zawierających dany termin. PostgreSQL 
także zawiera możliwość zbudowania indeksu odwróconego (np. GIN), jednak zdecydowano się na ElasticSearch z kilku powodów:
\begin{itemize}
    \item \textbf{Skalowalność horyzontalna} -- ElasticSearch został zaprogramowany z myślą o skalowalności. W celu uzyskania większej wydajności można dodać więcej 
    serwerów do danego klastra, a ElasticSearch automatycznie wydystrybuuje indeksy odwrócone.
    \item \textbf{Rozdzielenie odpowiedzialności} -- W przypadku zaimplementowania indeksu odwróconego w PostgreSQL, wyszukiwanie pełnotekstowe 
    dzieli te same zasoby co standardowe operacje bazodanowe. Skomplikowane wyszukiwanie pełnotekstowe może doprowadzić do spowolnienia działania bazy danych dla prostych operacji.
    \item \textbf{Wbudowana agregacja} -- ElasticSearch posiada wbudowany mechanizm agregacji, który pozwala na szybkie i łatwe tworzenie rankingów, bez obciążania głównej bazy danych.  
\end{itemize}   


    \chapter{Implementacja Systemu}
\label{ch:implementacja_systemu}

\section{Warstwa Prezentacji}

\section{Warstwa Logiki}
Backend zgodnie z projektem systemu, został zaimplementowany w języku Python, przy wykorzystaniu 
frameworku FastAPI. W przeciwieństwie do projektu systemu, gdzie opisywano podział logiczny, ten rozdział 
będzie opisywał organizację kodu, konfigurację serwera i mechanizmy scalające poszczególne komponenty systemu.

Kod źródłowy serwera jest zorganizowany w pakiety, odzwierciedlające podział na warstwy przyjęty podczas projektowania:
\begin{itemize}
    \item {\texttt{api/routes}} -- Zawiera definicję punktów końcowych, z których korzysta aplikacja kliencka.
    \item {\texttt{services}} -- Implementuje logikę biznesową aplikacji. Serwisy przetwarzają dane uzyskane od repozytoriów 
    i przekazują je do tras.
    \item {\texttt{repositories}} -- Warstwa dostępu do danych. Cały kod znajdujący się w tym pakiecie wykonuje bezpośrednie 
    zawołania do bazy danych, Elasticsearch lub MinIO w celu pobrania danych.
    \item {\texttt{api/schemas}} -- Przechowuje modele Pydantic (DTO) dla poszczególnych modułów.
    \item {\texttt{core}} -- Zawiera podstawowe konfiguracje systemu i często wykorzystywane funkcje pomocnicze.
    \item {\texttt{db}} -- Odpowiada za definicję modeli ORM oraz konfigurację połączenia z relacyjną bazą danych.
    \item {\texttt{external}} -- Zawiera pliki odpowiedzialne za inicjalizację i obsługę zewnętrznych systemów 
    (MinIO, ElasticSearch, Gemini).
\end{itemize}
Taka separacja pozwala na łatwe testowanie i utrzymanie kodu, ponieważ zmiany 
w logice biznesowej aplikacji, nie wymagają zmian w pakietach routes i repositories.
Każdy moduł systemu, zobrazowany na diagramie modułów funkcjonalności systemu 
(Rys \ref{rys:use_case_diagram}), posiada oddzielny router, serwis i repozytorium, 
co znacznie ułatwia poruszanie się po bazie kodu. 

Aplikacja do przechowywania wrażliwych danych konfiguracyjnych wykorzystuje bibliotekę pydantic-settings. 
Pozwala ona na walidowanie i wczytywanie wrażliwych danych (klucze API, sól, itp. ) z pliku .env do
klasy Settings. Dzięki temu aplikacja nie uruchomi się, jeżeli zabraknie któregoś z parametrów konfiguracyjnych 
w pliku .env.

Punktem wejściowym całego serwera jest plik main.py, w którym tworzony jest obiekt FastAPI. Podczas tworzenia instancji 
FastAPI, definiowany jest także cykl życia aplikacji. Pozwala on na zainicjalizowanie serwisów przed rozpoczęciem 
przyjmowania żądań od klienta i uruchomienie kodu czyszczącego po zakończeniu działania serwera. Szczególną rolę w tym procesie 
odgrywa zarządzanie połączeniami do poszczególnych źródeł danych. Tak jak wspomniano w rozdziale Projekt Systemu, zainicjalizowane 
zasoby są udostępniane komponentom systemu za pomocą mechanizmu wstrzykiwania zależności (Dependency Injection). W zależności od 
zasobu, przyjęto odmienne strategie zarządzania jego cyklem życia:
\begin{itemize}
    \item {\textbf{Relacyjna baza danych}} -- Do zarządzania połączeniem do relacyjnej bazy danych użyto mechanizmu sesji na żądanie 
    (session per request). Oznacza to, że każde zapytanie do serwera otrzymuje swoje własne połączenie do bazy danych, które wygasa po zwróceniu 
    wyniku do klienta.
    \item {\textbf{Wyszukiwarka pełnotekstowa}} -- Wyszukiwarki pełnotekstowe takie jak Elasticsearch posiadają wewnętrznego klienta 
    , który zarządza jego pulą połączeń HTTP. Do zarządzania jego cyklem życia stosuje się wzorzec Singleton. Oznacza to, że podczas inicjalizacji 
    aplikacji tworzona jest jedna wspólna instancja klienta ElasticSearch, która jest współdzielona przez wszystkie wątki i żądania. Dzięki temu aplikacja unika 
    kosztownego narzutu tworzenia nowego klienta dla każdego żądania.
\end{itemize}
Poniższy listing pokazuje implementację obu wyżej omówionych strategi. Funkcja \texttt{get\_db} wykorzystuje block try...finally
do bezpiecznego zamknięcia połączenia niezależnie od wyniku operacji. Z kolej funkcja \texttt{get\_es\_client} zwraca referencję do 
wcześniej zainicjalizowanego globalnego obiektu.

\begin{lstlisting}[language=Python,
	caption={Startegie zarządzania cyklem życia obiektu},
	label={lst:cykl_zycia}]
    # Session per request
    def get_db():
        db = SessionLocal()
        try:
            yield db
        finally:
            db.close()
    
    # Singleton
    def get_es_client():
        if es_client is None:
            raise RuntimeError(
                "Elasticsearch client is not initialized"
            )
        return es_client
\end{lstlisting} 


\subsection{Moduł Uwierzytelniania i zarządzania kontem}
Bezpieczeństwo aplikacji jest podstawą każdego systemu. Wszystkie mechanizmy powiązane z uwierzytelnianiem i zarządzaniem kontem 
znajdują się w serwisie \texttt{AuthService} i pakiecie \texttt{core.security}. W przeciwieństwie do prostych rozwiązań opierających 
się na przesyłaniu tokenów JWT w nagłówkach HTTP, zaimplementowane rozwiązanie wykorzystuje ciasteczka do przechowywania i przesyłania tokenów.
Proces logowania i utrzymywania sesji opiera się na dwóch tokenach:
\begin{itemize}
    \item \textbf{Access token} -- Krótkoterminowy token dostępu (ważny przez 5 minut) pozwalający na autoryzację użytkownika.
    \item \textbf{Refresh token} -- Długoterminowy token odświeżania (ważny przez 30 minut) służący do odnawiania sesji użytkownika, bez 
    konieczności ponownego logowania.
\end{itemize}
Oba tokeny nie są zwracane w ciele odpowiedzi (podatność na ataki XSS), lecz ustawiane są jako ciasteczka z flagą \texttt{HttpOnly}. 
Takie rozwiązanie ma swoje wady i zalety, ponieważ uniemożliwia one ukradnięcie tokenów przy pomocy ataku XSS, 
lecz powstają nowe zagrożenia takie jak np. atak Man-in-the-Middle (MitM) czy Cross-Site Request Forgery (CSRF). W celu zapobiegania tym 
atakom ciasteczko tworzone jest z jeszcze dwoma dodatkowymi flagami:
\begin{itemize}
    \item \textbf{Secure} -- Przeglądarka dołączy ciasteczko z tokenem tylko wtedy, gdy strona korzysta z protokołu HTTPS, dzięki czemu 
    ciasteczko zawsze będzie zaszyfrowane.
    \item \textbf{SameSite=lax} -- Ogranicza wysyłanie ciasteczka między stronami. W trybie lax przeglądarka blokuje przesyłanie ciasteczka 
    do żądań zainicjowanych przez zewnętrzne serwisy, ale zezwala na jego przesłanie podczas nawigacji najwyższego poziomu (np. kliknięcie w link nawigujący do
    mojej aplikacji).
\end{itemize}
Ustawienie SameSite na \texttt{lax} lub \texttt{strict} nie broni całkowicie przed atakami CSRF. 
W przypadku ustawienia SameSite na \texttt{lax} atakujący może wymusić wysłanie żądania z ciasteczkiem, 
inicjując nawigację najwyższego poziomu za pomocą prostego żądania GET. Ustawienie SameSite na \texttt{strict} 
komplikuje atak, ale nadal można go przeprowadzić poprzez przekierowania po stronie klienta lub podatności na subdomenach.
W celu większego zabezpieczenia przed tym atakiem zaimplementowano wzorzec Signed Double Submit Cookie. Mechanizm ten 
polega na podpisaniu tokenu zapisanego w ciasteczku CSRF przy użyciu sekretnego klucza serwera. Podczas weryfikacji żądania 
serwer najpierw sprawdza poprawność podpisu tokena w ciasteczku, a następnie porównuje jego zawartość z tokenem przesłanym w 
nagłówku HTTP.

Największym wyzwaniem w implementacji tego rozwiązania było połączenie asynchronicznej biblioteki walidującej CSRF (fastapi\_csrf\_protect) 
z synchronicznymi endpointami. Zdefiniowanie punktów końcowych jako asynchroniczne znacznie spowolniłoby działanie aplikacji, ponieważ serwer 
używa synchronicznego sterownika bazy danych i ElasticSearch'a, co doprowadziłoby do zablokowania  głównej pętli zdarzeń (Event Loop). Problem 
ten rozwiązano wykorzystując wzorzec wstrzykiwania zależności. Walidację tokenu CSRF wydzielono do asynchronicznej funkcji, która uruchamiana 
jest asynchroniczne przed właściwą logiką endpointu. FastAPI jest w stanie optymalnie zarządzać kontekstem wykonania: walidacja tokena uruchamia 
się na pętli zdarzeń, a logika endpointu jest delegowana do osobnej puli wątków.

\begin{lstlisting}[language=Python,
	caption={Asynchroniczna zależność w synchronicznym endpoincie},
	label={lst:cykl_zycia}]
    async def validate_csrf(
        request: Request, 
        csrf_protect: CsrfProtect = Depends()
    ):
        await csrf_protect.validate_csrf(request)

    @router.post("/materials", status_code=status.HTTP_201_CREATED)
    def create_material(
        material_data: MaterialCreate,
        _ : None = Depends(validate_csrf),
        db: Session = Depends(get_db)
    ):
        return material_service.create(db, material_data)
\end{lstlisting} 
 

\section{Opis wybranych modułów}    


	% \input{tekst/analiza}
	% \input{tekst/realizacja}
	% \input{tekst/podsumowanie}

    % Bibliografia - musi być, jest w pliku EE-dyplom.bib
    % Bibliography - must exist, is in the file EE-dyplom.bib
    \bibliografia

    % Strony końcowe - można zakomentować, jeśli zbędne
    % Additional pages - comment out if not needed
    
    % Wykaz symboli i skrótów - patrz opis w tekście przykładowym
    \acronymslist
    % Spis rysunków
    \listoffigures
    % Spis tabel
    % \listoftables
    % % Załączniki (plik appendices.tex)
    % \easyappendices
\end{document}
%%%%%%%%%%%%%%%%%%%%%%%%%%%%%%%%%%%%%%%%%%%%%%%%%%%%%%%%%%%%%%%%%%%%%%%%%%%

