\chapter{Wstęp}
\label{ch:wstep}

Proces nabywania wiedzy jest nieodłącznym elementem rozwoju każdego człowieka. Szybki postęp technologiczny w ostatnich dekadach znacząco zmienił 
powszechne podejście do edukacji. Nastąpił gwałtowny przeskok z papierowych, tradycyjnych nośników wiedzy na łatwo dostępne cyfrowe rozwiązania. 
Proces ten, określany mianem cyfrowej transformacji edukacji, wymusił poszukiwanie nowych, 
bardziej efektywnych metod zarządzania wiedzą i narzędzi do jej gromadzenia.

Współczesny uczeń musi być w stanie przyswoić dużą ilość wiedzy w ograniczonym czasie. Tradycyjne metody nauki, opierające się na tworzeniu 
monotonnych, ręcznie zapisanych notatek, mogą okazać się nieskuteczne w zależności od rodzaju przyswajanego materiału. W przypadku nauki 
słówek, definicji czy dat, coraz większą popularność zyskują metody dydaktyczne polegające na aktywnym powtarzaniu nabytej wiedzy. Jednym z 
najczęściej używanych i najskuteczniejszych sposobów jest nauka z samodzielnie przygotowanych fiszek. Manualny proces wykonywania fiszek ma 
jednak też swoje wady. Jest on wysoce czasochłonny, a nauka z wyciętych kartek jest niepraktyczna i niewygodna.

Odpowiedzią na te problemy jest cyfryzacja zestawów fiszek oraz dostarczenie narzędzi usprawniających ich zarządzanie. Głównym celem niniejszej 
pracy jest zaprojektowanie i zaimplementowanie nowoczesnej aplikacji internetowej, która eliminuje ograniczenia fizycznych zestawów fiszek.
Stworzony system oferuje zaawansowany edytor do tworzenia fiszek oraz możliwości kategoryzacji i współdzielenia zgromadzonej wiedzy.

Realizacja tak zdefiniowanego celu objęła pełny cykl wytwarzania oprogramowania (podejście End-to-End), od analizy wymagań i projektu systemu, 
poprzez implementację i przygotowanie środowiska uruchomieniowego oraz telemetrii. Rezultatem jest kompletny system oparty na modelu 
klient-serwer. Warstwa prezentacji została zaimplementowana jako aplikacja typu Single Page Application przy użyciu biblioteki React, 
natomiast serwer w języku Python przy wykorzystaniu frameworka FastAPI. Całość została skonteneryzowana w celu zapewnienia spójności środowiska 
oraz ułatwienia przyszłego skalowania i wdrażania aplikacji.

\section{Psychologiczne aspekty zapamiętywania}

Aby wyjaśnić, dlaczego stosowanie fiszek w edukacji jest tak wydajne, należy odwołać się do podstawowych badań z zakresu psychologii poznawczej. 
Kluczowe jest zrozumienie różnicy między biernym a aktywnym przyswajaniem wiedzy. Tradycyjne bierne metody dydaktyczne, polegające na 
wielokrotnym czytaniu notatek lub podręczników, często prowadzą do zjawiska iluzji kompetencji (ang. illusion of competence). 
Zjawisko to polega na fałszywym przekonaniu o umiejętności samodzielnego odtworzenia materiału z pamięci, po jedynie kilkukrotnym 
jego przeczytaniu \cite{brown2014make}. W rzeczywistości jednak cała wiedza często pozostaje jedynie w pamięci krótkotrwałej, czyniąc ją ulotną.

Fiszki pozwalają na zmianę strategii nauki na proces aktywny, w którym dochodzi do długotrwałej retencji wiedzy, eliminując problem 
pamięci krótkotrwałej. Podejście to polega na dwóch kluczowych mechanizmach kogniwistycznych:
\begin{itemize}
    \item \textbf{Aktywne przywołanie} -- Metoda nauki polegająca na próbie wydobycia informacji z pamięci poprzez testowanie swojej 
    wiedzy na każdym możliwym etapie nauki bez używania podpowiedzi. Fiszki nadają się do tego perfekcyjnie, ponieważ zawsze prezentują 
    tylko jedną stronę (awers), co zmusza użytkownika do samodzielnego sformułowania odpowiedzi przed jej weryfikacją (zobaczeniem rewersu).
    Badania pokazują, że ten wysiłek kognitywny zapewnia znacznie trwalszą retencję wiedzy niż proste wielokrotne studiowanie danego 
    materiału, co udowodniono w eksperymentach Roedigera i Karpicke \cite{roediger2006test}.
    \item \textbf{Metapoznanie i samoocena} -- Fiszki wymuszają na uczącym się ciągłą weryfikację stanu swojej wiedzy. W momencie 
    odsłonięcia rewersu karty, użytkownik dokonuje weryfikacji poprawności swojej odpowiedzi. Pozwala to na zidentyfikowanie 
    tematów, które nie zostały jeszcze w pełni opanowane i skierowanie na nie większej uwagi. Badania Kornella i Bjorka 
    wykazują, że zdolność do trafnej oceny własnej wiedzy jest kluczowa dla optymalizacji czasu nauki, lecz obarczona 
    ryzykiem błędu w samoocenie \cite{kornell2008optimizing}. Struktura fiszek minimalizuje ten błąd poprzez atomizację 
    materiału, tym samym ułatwiając trafną ocenę wiedzy na poziomie pojedynczych pojęć.
\end{itemize}

\section{Analiza istniejących rozwiązań}
Fiszki znacząco zyskały na popularności wraz z coraz większą dostępnością internetu i rozwojem technologicznym.
Na rynku istnieje wiele aplikacji oferujących szeroki zakres narzędzi służących do zarządzania i 
wspomagania procesu zapamiętywania. Wiele z nich opiera się na cyfrowej implementacji fiszek. 
W celu zdefiniowania wymagań projektowych dokonano analizy trzech głównych rozwiązań dostępnych na rynku:
Anki, Quizlet oraz Brainscape.

\subsection{Anki}
Anki to darmowe oprogramowanie typu open-source, które istnieje na rynku już od wielu lat. Projekt oferuje darmową aplikację 
desktopową oraz mobilną na system Android, lecz niestety płatną wersję na iOS. Jego główną zaletą jest 
implementacja zaawansowanego algorytmu powtórek w odstępach czasowych (Spaced Repetition System - SRS).
\begin{itemize}
    \item \textbf{Zalety:} Duża elastyczność w konfiguracji algorytmu SRS, obsługa wtyczek tworzonych przez społeczność i możliwość 
    uczenia się na wielu platformach offline.
    \item \textbf{Wady:} Wszystkie platformy posiadają bardzo nieintuicyjny i przestarzały interfejs graficzny. 
    Krzywa uczenia się obsługi programu jest wysoka, a edytor fiszek mimo dużej liczby funkcjonalności wymaga znajomości 
    HTML/CSS do zaawansowanego formatowania. Ponadto synchronizacja danych między urządzeniami nie jest natywna i wymaga 
    dodatkowej konfiguracji w serwisie AnkiWeb, a system organizacji materiałów jest mało czytelny.
\end{itemize}

\subsection{Quizlet}
Quizlet to komercyjne rozwiązanie, które zdobyło popularność dzięki intuicyjnemu interfejsowi i dostępności z poziomu 
przeglądarki oraz aplikacji mobilnej. Firma kładzie większy nacisk na implementację różnorodnych form nauki (np. poprzez 
gry) niż na surowe algorytmy powtórek.
\begin{itemize}
    \item \textbf{Zalety:} Nowoczesny i estetyczny interfejs użytkownika, ogromna baza publicznych zestawów fiszek stworzonych 
    przez społeczność Quizleta oraz wysoka dostępność na urządzeniach mobilnych i desktopowych bez konieczności instalacji oprogramowania.
    \item \textbf{Wady:} Model biznesowy typu freemium, polegający na przyciąganiu klientów poprzez oferowanie podstawowych funkcjonalności bezpłatnie, 
    jednocześnie wymagając płatnej subskrypcji za użytkowanie rozszerzonych funkcji systemu. Przykładem funkcjonalności premium jest możliwość wstawiania multimediów 
    do fiszek, czy nauka w trybie długoterminowym. Dodatkowo edytor treści fiszek jest uproszczony i nie pozwala na zaawansowane formatowanie tekstu lub wstawianie wielu zdjęć.
\end{itemize}

\subsection{Brainscape}
Brainscape jest dostępne jako aplikacja mobilna oraz webowa. Korzysta on z zaawansowanej metody metapoznania i samooceny do przyśpieszenia procesu nauki. 
W przeciwieństwie do standardowych algorytmów użytkownik zamiast wyboru odrzucenia/zostawienia fiszki, musi samodzielnie ocenić stopień znajomości odpowiedzi 
w skali od 1 do 5, co bezpośrednio wpływa na czas do następnej powtórki.
\begin{itemize}
    \item \textbf{Zalety:} Wymusza na użytkowniku ciągłą samoocenę, co znacznie przyśpiesza naukę. Aplikacja jest przejrzysta i prosta w obsłudze, 
    nastawiona na szybkie tworzenie prostych fiszek, a edytor treści pozwala na wstawianie plików dźwiękowych i graficznych. 
    \item \textbf{Wady:} Podobnie jak Quizlet, system opiera się na płatnych subskrypcjach, gdzie darmowa wersja posiada ograniczone funkcjonalności 
    (np. brak statystyk i możliwości wstawiania plików multimedialnych). Edytor treści nie oferuje funkcji tworzenia niestandardowych układów grafik i 
    bardziej zaawansowanego formatowania tekstu.
\end{itemize}

\subsection{Podsumowanie analizy}
Przeprowadzona analiza pozwoliła na zidentyfikowanie luki na rynku aplikacji do fiszek. Istnieje realne zapotrzebowanie 
na system, który będzie wspomagał użytkownika nie tylko w nauce, ale także w samym zarządzaniu zestawami fiszek, łącząc zalety 
powyższych aplikacji i eliminując ich główne wady. Tabela \ref{tab:porownanie_rozwiazan} ukazuje zestawienie porównawcze 
analizowanych systemów z aplikacją będącą przedmiotem niniejszej pracy.

\begin{table}[htbp]
    \centering
    \caption{Porównanie analizowanych rozwiązań z projektowanym systemem}
    \label{tab:porownanie_rozwiazan}
    \footnotesize % Zmniejszenie czcionki, aby tabela była czytelna
    \begin{tabularx}{\textwidth}{@{} p{2.2cm} X X X X @{}}
        \toprule
        \textbf{Cecha} & \textbf{Anki} & \textbf{Quizlet} & \textbf{Brainscape} & \textbf{Projektowana Aplikacja} \\
        \midrule
        
        \textbf{Model dystrybucji} & 
        Open Source (Desktop/Android), Płatny (iOS) & 
        Komercyjny (Freemium) & 
        Komercyjny (Freemium) & 
        \textbf{Open Source} \\ 
        \addlinespace % Dodatkowy odstęp dla czytelności
        
        \textbf{Dostępność} & 
        Desktop, Mobile (Offline first) & 
        Web, Mobile & 
        Web, Mobile & 
        \textbf{Web} \\ 
        \addlinespace
        
        \textbf{Interfejs (GUI)} & 
        Przestarzały, nieintuicyjny & 
        Nowoczesny & 
        Nowoczesny & 
        \textbf{Nowoczesny} \\ 
        \addlinespace
        
        \textbf{Organizacja danych} & 
        System talii (Decks) & 
        Listy i foldery (Płaska struktura) & 
        Klasy i talie & 
        \textbf{Pełny eksplorator plików (Zagnieżdżone foldery)} \\ 
        \addlinespace
        
        \textbf{Edytor treści} & 
        HTML/CSS (Wymaga wiedzy tech.) & 
        Uproszczony & 
        Podstawowy & 
        \textbf{Zaawansowany WYSIWYG (TipTap)} \\ 
        \addlinespace
        
        \textbf{Wsparcie AI} & 
        Brak (ew. zewnętrzne wtyczki) & 
        Płatne (Q-Chat) & 
        Brak & 
        \textbf{Automatyczne tagowanie (Gemini API)} \\ 
        
        \bottomrule
    \end{tabularx}
\end{table}

Projektowana aplikacja ma za zadanie wypełnić zidentyfikowaną na rynku niszę. Implementowane rozwiązanie webowe charakteryzuje się 
nowoczesnym i intuicyjnym interfejsem graficznym, zapewniającym duży komfort pracy. Kluczowym elementem jest zaawansowany edytor tekstu, 
który pozwala na tworzenie niestandardowych układów obrazów oraz bogate formatowanie tekstu bez konieczności znajomości kodu HTML czy CSS.
Oprócz tego system kładzie duży nacisk na aspekty społecznościowe, takie jak komentarze czy system głosowania, przekształcając samotny 
proces nauki w doświadczenie oparte na współpracy i wymianie wiedzy. Całość dopełnia łatwy w obsłudze eksplorator plików z obsługą 
tworzenia zagnieżdżonych folderów, co ułatwia użytkownikom centralizację wiedzy i zarządzanie wszystkimi materiałami. 